%% compilation command: latex -file-line-error -interaction=nonstopmode <file>
\documentclass{article}
\usepackage{algorithm2e}[longend noline]
\usepackage{float}
\usepackage{listings}
\lstset{escapeinside={\%*}{*)}, language=Ada}

% we'd like to use 'hyperref' for clickable citation links in PDF, but
% that conflicts with 'listings \lstinline'

\title{Robust error correction in a generalized LR parser}
\author{Stephen Leake}

\newcommand{\code}[1]{`\lstinline|#1|'}

\begin{document}
\section{Abstract}

\section{Introduction}
Emacs Ada mode has used an LR parser to support indentation, syntax
highlighting, and navigation since 2013 \cite{Emacs Ada mode news}.
However, the parser did not provide error correction, so indentation
was often confusing when the syntax was incorrect, as it usually is in
an interactive editing environment. This motivated the search for
error correction algorithms.

Grune \cite{Grune 2008} provides a thorough overview of error
correction algorithms in LR and LL parsers. Of those, McKenzie
\cite{McKenzie 1995} provides the foundation for the current work.

The McKenzie algorithm works by exploring the parse table (or
Deterministic Parsing Automata (DPA) as \cite{McKenzie 1995} calls it)
at the error point, finding tokens to insert. It also tries deleting
tokens following the error point. Each possible solution, together
with the parse stack at the error point, forms a
\textit{configuration}. Each configuration also has a cost, determined
by what tokens are inserted and deleted. At each step in the
algorithm, new configurations are generated from the current error
point. Then the minimum cost configuration is checked to see if it
succeeds; if not, more configurations are generated.

There are several situations where the McKenzie algorithm can take a
long time, or is inefficient.

For example, consider this Ada code:
\begin{lstlisting}
procedure Example_01
is begin
   Msg : constant String;
begin
   Put_Line (Msg);
end;
\end{lstlisting}

There is an extra \code{begin} immediately after \code{is} (a common
occurance while editing code). However, the error is not detected
until \code{:} after \code{Msg}, which can only occur in declarations,
not statements.

To fix this, the McKenzie algorithm must insert\\ \code{; end; begin
  IDENTIFIER}, or delete \code{constant String; begin} and then insert
\code{;}. A much better solution would be to ``push back'' \code{begin
  Msg}, and then delete \code{begin}.

A harder problem is when there are several missing ``end''s, because
the user is typing a nested statement:
\begin{figure}[H]
\begin{lstlisting}
begin
   if A then
      B;
      if C then
         loop
           Do_Something;
end;
\end{lstlisting}
\caption{missing many 'end'}
\label{ex:min_com_if_if_loop}
\end{figure}

Here we are missing several tokens: \code{end loop; end if; end if;}.
McKenzie will theoretically find the solution that inserts all of
these, but along the way it will try inserting every possible
statement as well, wasting a lot of time, and in practice hitting a
time-out limit. This article introduces the \code{Minimal\_Complete}
algorithm, which quickly finds the minimal number of tokens to insert
to complete the trailing statement in situations like this.

This article also introduces several other new operations for the core
McKenzie algorithm to try at the error point, and adapts the algorithm
to work with a generalized parse.

Unless quoting from another source, we use the notation described in
\cite{DeRemer 1982}.

%% copy initial parts of mckenzie to define 'config' etc? Also dragon
%% to define "parse state", "kernel"

\section{The parsing context}
\subsection{Generalized parsing}
We use a generalized LR parser \cite{Tomita 1986}, to tolerate
conflicts in the parse table. This allows using the Ada grammar as
given in the ISO language standard appendix P \cite{Ada 2012}, which
is not LR(1). Using that grammar directly ensures we are implementing
the correct language, and simplifies updating the parser to a new
language version.

This means the main parser may have several parallel parsers executing
when an error is encountered. The McKenzie error correction algorithm
is enhanced to maintain state information for each parser. When more
than one solution is found with the same cost, additional parsers are
created to use them.

The parser builds a syntax tree; to handle parallel parsers, we use a
branched tree similar to Tomita's sub-tree sharing. The syntax tree
will contain ``virtual tokens'' that are inserted by error correction.

The entire input token sequence is kept in memory, to allow arbitrary
push back.

\subsection{Partial parse}
In order to handle very large files, the parser supports ``partial
parse''; parsing only part of a file.

Note that this is not the same as ``incremental parse'', where an
existing syntax tree from a previous parse is modified based on a text
edit. Integrating this error correction algorithm with incremental
parse is the subject of future work.

In order to minimize the amount of text passed, we modify the grammar
to allow smaller chunks of code to be accepted as a complete parse. In
Ada, this means adding \code{declaration} and \code{statement} to
\code{compilation\_unit}.

When the file length is greater than a threshold, Emacs invokes a
partial parse whenever a parse is needed. To do this, it first uses a
regular expression search to find a reasonable start point, then finds
a possible matching end that includes the requested parse position, and
passes that region to the parser.

In Ada, the search for a start point finds a block begin, or the point
after a block end. If a block begin was found, the matching end is
looked for; otherwise, the requested parse point is the end point.

This imposes a greater burden on the error correction algorithm, since
the parsed text will often be incomplete.

\section{Extensions to McKenzie}
We define the following operations that are tried in each McKenzie
step:
\begin{itemize}
\item \code{push\_back}
\item \code{undo\_reduce}
\item \code{Try\_Insert\_Quote}
\item \code{Minimal\_Complete}
\item \code{Language\_Matching\_Begin\_Tokens}
\item \code{Language\_Fixes}
\end{itemize}

\subsection{push\_back}
\code{push\_back} pops the top parse stack item, and moves the input
stream pointer back to the first terminal contained by that item. We
call the point in the input stream at which insert and delete is done
the ``edit point''; it may not be an error point. \code{push\_back}
moves the edit point.

\subsection{undo\_reduce}
\code{undo\_reduce} undoes the reduce that produced the top stack item
(which must be a nonterminal), replacing the top stack item by the
sequence of stack items just before the reduction, without
moving the edit point. This requires a syntax tree that records the
shifts and reductions done during the parse. The edit point is not
moved. This operation serves two purposes;
\begin{enumerate}
\item It allows a subsequenct \code{push\_back} to push back fewer tokens.

\item It allows token insertions that would otherwise be forbidden by the
grammar.
\end{enumerate}

To illustrate the second point, consider:
\begin{figure}[H]
\begin{lstlisting}
procedure Example_2
is
   I : Integer;
begin
   procedure Put_Top_10
   is begin
   ...
   end Put_Top_10;
begin
end Example_2;
\end{lstlisting}
\caption{}
\label{ex:extra_begin_2}
\end{figure}
There is an extra \code{begin} after
\code{I : Integer}. The error is detected at \code{procedure}; at that point,
the parse stack looks like (top is to the left):
\small{
\begin{verbatim}
245 : BEGIN, 208 : declarative_part, 159 : IS, 36 : subprogram_specification, 0 :
\end{verbatim}
}
Here the numbers label the states, terminals are in uppercase,
nonterminals in lowercase.

The grammar productions relevant to this example are:
\begin{verbatim}
declarative_part <= declarations |  ;
declarations <= declarations declaration | declaration ;
declaration <= subprogram_declaration | ... ;
\end{verbatim}

Fixing the parse error starts by \code{push\_back BEGIN, delete BEGIN},
leaving \code{declarative\_part} on top of the stack. \code{procedure}
is the next token, which is illegal in state 208; it starts a
\code{subprogram\_declaration}, but we've already ``closed'' the
declaration section by reducing to \code{declarative\_part}. We could
do \code{push\_back declarative\_part}, but that moves the edit point to
before the object declaration for \code{I}, where there is no error
and nothing helpful to insert or delete. Instead, \code{undo\_reduce}
leaves the stack as:

\begin{verbatim}
137 : declarations, 159 : IS, 36 : subprogram_specification, 0 :
\end{verbatim}
and now \code{procedure} is legal.

We do not maintain a syntax tree for the parsing done during error
correction; doing that proved to be much too slow. Therefore the
\code{undo\_reduce} operation can only be applied to configurations
where the top stack item was produced by the main parse, so it has a
valid syntax tree entry. An exception is when the nonterminal is
empty; that is easy to undo.

\subsection{Try\_Insert\_Quote}
Missing string quotes cause problems for the McKenzie algorithm.
Consider the code:
\begin{lstlisting}
A : String := Now is the time for all good men";
\end{lstlisting}
There is a missing quote before \code{Now}. In Ada, strings cannot
cross newline, and the lexer handles the error by inserting a virtual
quote just before the existing one. Then the parser sees a list of
identifiers followed by an empty string literal. The McKenzie
algorithm would have to delete all the identifiers one by one, with a
cost for each.

The \code{Try\_Insert\_Quote} operation attempts to find a better
place to insert the string quote, depending on the relative placement
of the unbalanced quote and the parse error.

\begin{itemize}
\item If the parse error is at the unbalanced quote, assume the unbalanced
quote is the intended closing quote, and insert the opening quote one
non-empty token before it. Example:
\begin{lstlisting}
   A := "for all" & good ";
\end{lstlisting}
We are in the process of splitting a string across lines; we just
added \code{" \&}, but are missing the \code{"} before \code{good}.
This solution inserts that missing quote.

\item If the parse error is after the unbalanced quote, assume the unbalanced
quote is the intended opening quote, and insert the closing quote at
the line end. Example:
\begin{lstlisting}
   A := "for all" & "good ;
\end{lstlisting}
The missing \code{"} should be after \code{good}.
This solution inserts that missing quote.

\item If the parse error is before the unbalanced quote, assume the unbalanced
quote is the intended closing quote, and insert the opening quote:

\begin{itemize}
\item before the error token. Example:
\begin{lstlisting}
   A := for all good";
\end{lstlisting}
The missing \code{"} should be before \code{for}. The parse error is at \code{all};
this solution inserts the missing quote before \code{all}, which is
almost right.

\item One non-empty token before the unbalanced quote. Same example,
  but this inserts the \code{"} before \code{all}, which is correct.

\item If there is a string literal on the parse stack, assume the
  closing quote of that string literal is new (or extra), push back
  thru the token containing that string literal, and extend the string
  literal to the unbalanced quote.
  Example:
\begin{lstlisting}
   A := "for all" good";
\end{lstlisting}
The \code{"} after \code{all} is extra. The parse error is at \code{good};
this solution in effect deletes the extra quote.

Note that the search for a string literal on the parse stack must take
into account nonterminals that may contain a string literal. The set
of nonterminals that may contain a string literal is provided as a
function call written by the grammar author; it is not computed from
the grammar because it should not include higher level nonterminals
that are not likely to be contained in a string; for Ada, it stops at
expression.
\end{itemize}
\end{itemize}

Since the lexer recognizes string literals, we cannot actually insert
an unbalanced string quote; we actually delete all tokens between the
inserted quote and the unbalanced quote, which matches what the lexer
would have returned. This is still much better than the original
McKenzie algorithm, because we do all the deletions in one step, with
one low cost.

\subsection{Minimal\_Complete}
The \code{Minimal\_Complete} strategy is to insert the minimum number
of tokens to finish the current grammar production. This is supported
by precomputing a set of \code{Minimal\_Complete\_Action} for each
parser state.

Section \ref{minimal-complete-compute} gives the algorithm for
computing the\\ \code{Minimal\_Complete\_Actions}; here we describe how
they are used in error correction.

Each \code{Minimal\_Complete\_Action} is either ``insert this terminal
token'' or ``reduce to this nonterminal token''.

Consider the code in figure \ref{ex:min_com_if_if_loop}. The best error
correction is to provide all of the missing ``ends''.

When parsing this code, an error is detected at the final \code{;};
the parser is expecting \code{end loop;}. The kernel of that state has
one production:
\begin{verbatim}
loop_statement <= LOOP sequence_of_statements END ^ LOOP
   identifier_opt SEMICOLON
\end{verbatim}
where the carrot (\verb|^|) shows the parse point (also known as
``dot''). In this case, it is easy to see that
\code{Minimal\_Complete\_Action} must be ``insert \code{loop} ''.
Similarly, after parsing \code{loop}, \code{Minimal\_Complete\_Action}
is ``reduce to \code{identifier\_opt} '', and then ``insert \code{;}
''. After that, we will be completing the inner \code{if then}
statement, and then the outer \code{if then} statement. At that point,
the existing \code{end ;} is legal.

To handle cases where only the leading part of a nonterminal
production needs to be inserted, the check to see if the original
error token is now legal must be performed after each token is
inserted; this means that \code{Minimal\_Complete} inserts at most one
token for each cycle of the underlying McKenzie algorithm.

In order to prefer the \code{Minimal\_Complete} solution over others,
we give it a negative cost. In figure \ref{ex:min_com_if_if_loop}, the
grammar is a small subset of Ada, the
default insert and delete cost is 4, delete \code{begin} is cost 1,
delete \code{END}, \code{;} are cost 2. With \code{Minimal\_Complete}
cost 0, -1, or -2, no solution for figure \ref{ex:min_com_if_if_loop}
is found, with an enqueue limit of 120,000. With
\code{Minimal\_Complete} cost -3, the desired solution is found with
cost 9, after enqueueing 6804 configurations and checking 1051.

As usual, there is a trade off here; sometimes other solutions would
be better than \code{Minimal\_Complete}. For example, consider:
\begin{lstlisting}
for I in 1 to Result_Length loop
end loop;
\end{lstlisting}
Here \code{to} should be \code{..} (this is an actual error the author
typed late one night). The error is detected at \code{to}, so the
desired solution is\\ \code{delete IDENTIFIER, insert ..}. With the
\code{Minimal\_Complete} cost set to 0, this solution is found, along
with 19 other solutions with the same cost; a few of these change the
code to:

\begin{lstlisting}
in to.Result_Length loop
in 1 .. to*Result_Length loop
in 1 .. to/Result_Length loop
\end{lstlisting}

Finding these takes a relatively long time; 1273 configurations
were enqueued and 218 checked.

Setting \code{Minimal\_Complete} cost to -1 finds similar solutions,
but more quickly (341 enqueued, 56 checked); \code{insert ..} is done
by \code{Minimal\_Complete}, so it is cheaper, and more expensive
solutions are not checked.

Setting \code{Minimal\_Complete} cost to -3 (as required for
figure \ref{ex:min_com_if_if_loop}) finds one cost 3 solution very
quickly (67 enqueued, 10 checked); it changes the code to:
\begin{lstlisting}
for I in 1 .. to loop Result_Length; loop
end loop;
\end{lstlisting}
which causes another syntax error later in the code (missing
\code{end loop;}). Here three tokens were inserted by
\code{Minimal\_Complete}; \code{.. loop ;}.

In the production Ada parser, -3 proves to be a good compromise for\\
\code{Minimal\_Complete} cost.

In some cases, the action required for \code{Minimal\_Complete} is
reduce, not shift. For example:
\begin{lstlisting}
case Current_Token is
= +Right_Paren_ID then
   Matching_Begin_Token := +Left_Paren_ID;
else
   Matching_Begin_Token := Invalid_Token_ID;
end if;
\end{lstlisting}
%% minimal_complete_convert_if_to_case.ada_lite
Here the user is in the middle of converting an \code{if} statement to
a \code{case} statement. The relevant grammar productions are:
\begin{verbatim}
if_statement
  : IF expression_opt THEN sequence_of_statements
    elsif_statement_list ELSE sequence_of_statements
    END IF SEMICOLON ;
case_statement
  : CASE expression_opt IS case_statement_alternative_list
    END CASE SEMICOLON ;
\end{verbatim}
An error is detected at \code{=}; \code{Minimal\_Complete} inserts\\
\code{when NUMERIC\_LITERAL =>}, then the original McKenzie algorithm
inserts \code{if NUMERIC\_LITERAL}, which makes the remaining code
legal, terminating one error correction session. Then parsing proceeds
to then end of the input, where another error is enountered; missing
\code{end case;}. At that point, the parse state kernel has one
production:
%% ada_lite_lr1.parse_table, state 2891
%% 98.1:
%% reduces to
\begin{verbatim}
if_statement <= IF expression_opt THEN sequence_of_statements
   ELSE sequence_of_statements END IF SEMICOLON ^;

Minimal_Complete_Action => if_statement
\end{verbatim}
Since the \code{Minimal\_Complete\_Action} token is a nonterminal, the action
is reduce, not shift. That leads to several more states where the\\
\code{Minimal\_Complete\_Action} is reduce:
%% states 1711, 1708
\begin{verbatim}
compound_statement <= if_statement ^
Minimal_Complete_Action => compound_statement

statement <= compound_statement ^
Minimal_Complete_Action => statement

  ...
\end{verbatim}

and finally arrives at the
state kernel:
%% state 370
\begin{verbatim}
case_statement <= CASE expression_opt IS
   case_statement_alternative_list ^ END CASE SEMICOLON
case_statement_alternative_list <=
   case_statement_alternative_list ^ case_statement_alternative

Minimal_Complete_Action => END
\end{verbatim}
which inserts \code{end}. \code{Minimal\_Complete} does all required
reductions, and one insertion, in one McKenzie step.

If there is more than one production in the kernel for a parse state
that gives the minimum length for that state, there is more than one\\
\code{Minimal\_Complete\_Action} for that state. In that case, we look
at the length after the parse point for each production in the kernel
for the state that the shift or reduce goes to; if one of the actions
results in a minimum length, that action is chosen. If more than one
action gives the minimum length, all are kept. Therefore
\code{Minimal\_Complete} maintains a queue of configurations with an
action to check. Each may produce a new configuration for the next
McKenzie step.

Sometimes the minimal length production cannot be computed at compile
time. Consider the Java code fragment:
\begin{figure}[H]
  \small{
    \begin{lstlisting}
{ B = UpdateText (A }
    \end{lstlisting}
    }
\caption{}
\label{ex:recursive_length_after_dot}
\end{figure}
% minimal_complete_recursive_length_after_dot.java_exp_ch19
% minimal_complete_recursive_length_after_dot.parse
This is missing \code{);} after \code{A}. The error is detected at
`\lstinline|}|'. At that point, the kernel is:
%% java_expressions_ch19_lr1.parse_table state 21
\begin{verbatim}
LambdaExpression <= Identifier ^ MINUS_GREATER Identifier
LeftHandSide <= Identifier ^
ClassType <= Identifier ^
MethodInvocation <= Identifier ^ LEFT_PAREN ArgumentList RIGHT_PAREN

Minimal_Complete_Action => (MINUS_GREATER, LeftHandSide, ClassType)
\end{verbatim}
Here two actions are reduce, and therefore there are no tokens in
these productions after the parse point, so we have to look at the
next state to find the minimal length production. Reducing to
\code{LeftHandSide} goes to the state:
%% java_expressions_ch19_lr1.parse_table state 26
\begin{verbatim}
Assignment <= LeftHandSide ^ EQUAL Expression
\end{verbatim}
where there are two tokens needed to complete the production.

Reducing to \code{ClassType} goes to the state:
%% java_expressions_ch19_lr1.parse_table state 32
\begin{verbatim}
PostfixExpression <= ClassType ^
ClassType <= ClassType ^ DOT Identifier
\end{verbatim}
Once again, we reduce \code{ClassType} to the next state, and the
next, until we find a shift. This ends in the state:
%% states 30 29 28 27 24 22 83
\begin{verbatim}
MethodInvocation <= Identifier LEFT_PAREN ArgumentList ^ RIGHT_PAREN
ArgumentList <= ArgumentList ^ COMMA Expression
\end{verbatim}
which has 1 token after the parse point; this is the minimal length.
All of these reductions, and the final shift, are done in one McKenzie
step. The function that computes the length after parse point is
recursive, and explores all paths that might be minimal.

Sometimes \code{Minimal\_Complete\_Actions} results in an ambiguous
parse. Consider the following Java code:
\begin{figure}[H]
  \small{
\begin{lstlisting}
{ UpdateText (A) }
\end{lstlisting}
}
\caption{}
\label{ex:minimal_complete_ambiguous}
\end{figure}
% minimal_complete_ambiguous.java_exp_ch19
In the subset of Java we are using for this example, this is not a
complete statement. The relevant productions are:
\begin{verbatim}
StatementExpression     <= PostIncrementExpression | PostDecrementExpression ;
PostIncrementExpression <= PostfixExpression '++' ;
PostDecrementExpression <= PostfixExpression '--' ;
PostfixExpression       <=
   ClassType | MethodInvocation | PostIncrementExpression | PostDecrementExpression ;
ClassType               <= Identifier | ClassType '.' Identifier ;
\end{verbatim}
To complete the code fragment in figure
\ref{ex:minimal_complete_ambiguous}, we have to insert either
\code{++} or \code{--}. The error is detected at the closing brace;
the state is:
% java_expressions_ch19_lr1.parse_table state 61
\begin{verbatim}
PostIncrementExpression <= PostfixExpression ^ PLUS_PLUS
PostDecrementExpression <= PostfixExpression ^ MINUS_MINUS
Minimal_Complete_Action => (PLUS_PLUS, MINUS_MINUS)
\end{verbatim}
There are two actions, neither reduce, so we enqueue both. The next
McKenzie step inserts \code{;} in each, completing the error recovery
session with two solutions. In the main parser, both solutions parse
to end of input. In the absence of error correction multiple parallel
parsers getting to end of input is an error (ambiguous parse), but
since error correction often produces multiple solutions, we don't
report an error, one solution is chosen arbitrarily to be the final
parse.

\subsection{Matching\_Begin}
Consider the Ada code fragment:
\begin{lstlisting}
procedure ... end;

   end if;
end Foo;
\end{lstlisting}
This is a result of cut and paste.
\code{procedure Foo is begin if expression then} is missing before
\code{end if}.

The error is detected at the \code{end} in \code{end if}.
\code{Minimal\_Complete} is no help here; the parse stack looks like:
\begin{verbatim}
34 : subprogram_body, 0 :
\end{verbatim}
\code{Minimal\_Action} in state 34 is reduce to
\code{compilation\_unit}, which is then reduced to
\code{compilation\_unit\_list}, which is the grammar start symbol,
which has no minimal complete action. So \code{Minimal\_Complete} has
nothing useful to insert.

To the grammar author, the solution is obvious. We capture that
knowledge by having the grammar author provide a function\\
\code{Language\_Matching\_Begin\_Tokens}, which takes the current parse
stack and the next three input tokens as input, and returns a list of
tokens to try inserting at the edit point.

The \code{Language\_Matching\_Begin\_Tokens} function is similar to the
table\\ $E(A,a)$ given in \cite{FMQ 1980} for an LL parser:
\begin{equation}
E(A,a) \equiv x \, | \, A \Rightarrow^* xay, \, \mathtt{Cost(x)\; is\; minimized}
\end{equation}
However, we are using an LR parser, so that table is not directly
applicable. We could try to compute a table that gives the minimal
sequence of tokens to insert starting in any state $s$ , and
allowing any token $a$ as the next token:
\begin{equation}
M(s,a) \equiv y \, | \, A \Rightarrow^* xyaz, \, A \in Kernel(s), \, \mathtt{Cost(x)\; is\; minimized}
\end{equation}
where $x$ is the prefix of the production $A$ in state $s$.
That is the table that the McKenzie algorithm computes on the fly; it
is not worth precomputing. It is worth providing
\code{Language\_Matching\_Begin\_Tokens} to shortcut some common
situations.

In Ada, three tokens are required to determine the proper match for
\code{end}; consider:
\begin{lstlisting}
case is end case;
loop end loop;
block_1 : begin end Block_1;
package Parent_1.Child_1 is begin end Parent_1.Child_1;
\end{lstlisting}
In the case and loop statements, the three tokens starting with
\code{end} are \code{end loop ;} and \code{end case ;}; here we only
need two tokens to determine that the matching begin is \code{case} or
\code{loop}.
To distinguish between the named block statement and the package
declaration, we need three tokens; in a named block statement the name
must be a simple identifier, with no dots, so the third token must be
\code{;}.

In Ada, \code{Language\_Matching\_Begin\_Tokens} returns the proper
statement start token for \code{end ... }, and for
\code{then, else, elsif, exception}. For \code{when}, it returns
\code{case IDENTIFIER is}, which assumes a partial case statement is
more common than a partial exception handler. For any other error
token, it returns an empty token list; no guess is better than a bad
guess.

\code{Language\_Matching\_Begin\_Tokens} also returns a flag\\
\code{Forbid\_Minimal\_Complete}, which is True when
\code{Minimal\_Complete} would be harmful. In Ada, this is set True
when the error point is after \code{end} in one of the \code{end}
sequences above; it is better to push back \code{end}.

\subsection{Language\_Fixes}
To take advantage of the redundant block name information in Ada, we
provide a general hook \code{Language\_Fixes}; it takes as input all of
the current error recovery state, and enqueues new configurations to
test.
Consider:
\begin{lstlisting}
procedure Proc_1
is begin
    Block_1:
    begin
end Proc_1;
\end{lstlisting}
Here \code{end Block\_1;} is missing. The grammar rules for Ada do not
require the start and end block names to match; that is checked later
in the compilation process. To take advantage of it for error
correction, we add that check as a parse-time action in the grammar
declaration:
\begin{verbatim}
block_statement
  : block_label_opt BEGIN handled_sequence_of_statements END
    identifier_opt SEMICOLON
    %()%
    %(return Match_Names
        (Lexer, Descriptor, Tokens, 1, 5, End_Name_Optional);)%
  ;
\end{verbatim}
Here the first `\verb|%()%|' gives the post-parse action, which is run
after the parse is complete and the syntax tree is available; Emacs
uses this action to compute indent, navigation, and highlight. The
second `\verb|%()%|' gives the in-parse action, which is run when the
production is reduced, both in the main parse and during error
correction.

Here \code{Match\_Names} will return a status of
\code{Match\_Names\_Error} if the names do not match, with the error
point after the final \code{;}, and the production not reduced (so it
is possible to edit the token sequence without an
\code{undo\_reduce}). \code{Language\_Fixes} then tries to determine
the best fix based on the name information.

In this example, the check fail is reported after \code{end Proc\_1;}.\\
\code{Language\_Fixes} finds the matching \code{procedure Proc\_1} on
the parse stack, and inserts \code{end Block\_1;} before
\code{end Proc\_1;}.

Consider:
\begin{lstlisting}
package Parent_1.Child_1
is begin
    begin
end Parent_1.Child_1;
\end{lstlisting}
Here we might expect \code{Match\_Names} will fail with
\code{Extra\_Name\_Error}, but instead we get a parse error on \code{.}
in \code{Parent\_1.Child\_1}; block names must be simple identifiers.
Since this is similar to \code{Match\_Names\_Error},
\code{Language\_Fixes} handles it, finding the matching name; the
fix is to insert \code{end ;} before\\ \code{end Parent\_1.Child\_1;}.

Consider:
\begin{lstlisting}
Block_1 :
begin
    if A then
       null;
end Block_1;
\end{lstlisting}
Here we get a syntax error on the final \code{Block\_1}; \code{if} is
expected. Again, \code{Language\_Fixes} handles it, finding the
matching name.

More complex patterns can be recognized in \code{Language\_Fixes}; see
the production Ada code.

\section{Computing Minimal\_Complete}
\label{minimal-complete-compute}

\subsection{Grammar recursions}
We first compute the recursions in the grammar. A `{\it recursion}' is
the result of a cycle in the grammar productions; for example:
\begin{verbatim}
association_list
  : association_list COMMA association_opt
  | association_opt
  ;
\end{verbatim}
The first right hand side (RHS) is direct left recursive; the second
is not recursive.

Recursion can also be indirect; consider:
\begin{verbatim}
name
  : IDENTIFIER
  | selected_component
  ;
selected_component : name DOT IDENTIFIER ;
\end{verbatim}
Together, \code{name, selected\_component} are indirect recursive.

We can form a graph representing the grammar by taking the
nonterminals as the graph vertices, and if there is a production $A
\Rightarrow xBy$ then there is a directed edge from the $A$ to $B$.
Then recursion is represented by a cycle in the graph.

In a useful grammar, every recursion must have a non-recursive
RHS in one of the nonterminals, to terminate the recursion.

We use Johnson's algorithm \cite{graph-cycles} to find the cycles in
the graph. However, that algorithm does not apply to `{\it
  multigraphs}', which have more than one edge connecting any two
nodes. Real grammars can be multigraphs, so we first filter out all
such edges, and add them back after we compute the cycles.

Some languages have a lot of recursion, so it can be prohibitive to
compute the exact set of cycles in the graph. For example, Java SE 12
(using the grammar given in chapter 19 of the language reference
manual \cite{javarm}) takes too long to compute. In that case, we only
compute the strongly connected components (SCCs) (which is one of the
steps in \cite{graph-cycles}), and use that as the recursion. This
gives more recursion than necessary, and thus reduces the number of
minimal actions computed, but still gives good performance in error
correction. For Ada 2012, computing the full recursion is fast, but
for Ada 2020, it is prohibitively slow.

Once we have the cycles or SCCs, we add a Boolean \code{recursive}
flag to the items in the state kernels; true if the production is in a
cycle or SCC and the recursive token is the first token (ie the
production is left recursive, direct or indirect), false otherwise.

%FIXME; example of why non-left recursion is ok;
               %%  ada_lite_lalr_re2c_t1.parse_table - not listed in state order here, to
               %%  State 77:
               %%       57.0:actual_parameter_part <= LEFT_PAREN ^ association_list RIGHT_PAREN ; ( 2 => Other)
               %%      103.0:name <= name LEFT_PAREN ^ range_list RIGHT_PAREN ; ( 1 => Direct_Left,  3 => Other)
               %%
%% In State 77, Dot_ID is association_list which has
%%                    Other recursion; we say "there is recursion at the parse point".
%%                --  However, Length_After_Dot is correct; it assumes the
%%                --  recursion-breaking case for the expansion of association_list. So
%%                --  this is the same as no recursion at the parse point
\subsection{Other preliminaries}
Next, we compute the minimal terminal token sequence for each production.
This is the same as $S(A)$ from FMQ \cite{FMQ 1980}:
\begin{equation}
S(A) \equiv x \in V_t^* \, |\, A \Rightarrow^* x, \, \mathtt{Cost(x)\; is\; minimized}
\end{equation}
where the cost of each token is 1. Including token costs at this point
would make it very difficult to assign useful costs; we only use token
costs in the run-time portion of the algorithm.

Next we need \code{Minimal\_Terminal\_First(A)}, which is the first
token in $S(A)$, or the invalid token $\xi$ if $S(A)$ is empty.

Next we need $Nullable(A, \omega)$, which gives the production that
reduces $A$ to $\epsilon$:
\begin{equation}
Nullable(A, \omega) \equiv (B, \psi) \in P\, |\, (A \Rightarrow B \gamma, B \Rightarrow^* \epsilon)\ else\ \xi
\end{equation}

\subsection{Minimal Complete Action}
For each state, we consider each LR1 item in the kernel. There are
several possible cases, listed in priority order:
\begin{enumerate}
  \setcounter{enumi}{-1}

  %case 0
\item There is only one item in the kernel. Recursion is
  ignored, minimal action is given by \code{Compute\_Action (Dot)},
  where \code{Dot} is the token after dot in the kernel item.

  %case 1
\item Dot is at the end of the production; the actual production
  length must be computed at runtime. Recursion is ignored; the
  minimal action is reduce to the item LHS.

  %case 2
\item The item is left recursive; there is no minimal action.

  %case 3
\item There is no recursion, and dot is not at the end of the
  production. If the number of tokens after dot is minimal within the
  kernel, include \code{Compute\_Action (Dot)} in the minimal actions.

\end{enumerate}
\begin{lstlisting}
function Compute_Action (Token)
   if Token in Terminals then
      return Action (State, Token);

   else
      if Minimal_Terminal_First (Token) = Invalid_Token then
         return (Reduce 0 tokens to Token);

      else
         return Action (State, Minimal_Terminal_First (Token));
      end if;
   end if;
end Compute_Action;
\end{lstlisting}
where \code{Action (State, Token)} returns the parse action for
the token in the state.

\section{Implementation}
%% optimize data structures for speed; fixed length queue, stack;
%% fibonacci heap
%% multiple threads
%% LR1 vs LALR

\section{Results}
This error correction algorithm has been in production use in Emacs
ada-mode since November 2018. The parser can be configured to output
information about each error recovery; that is summarized in figures
\ref{table:full-recover-stats}, \ref{table:partial-recover-stats} for one
month of the author's use. The enqueue limit is set at 58\_000; there is a
noticeable delay when that limit is hit. The maximimum enqueu value is
higher because multiple tasks are used in error recovery; each task is
allowed to finish its current operation before aborting due to the limit.

This shows that the minimal complete operation is used in a large
majority of cases, along with insert.

\begin{figure}[H]
\begin{tabular}{| l | r | r |}
\hline
fail enqueue limit    & 1126     & 3\%     \\
\hline
ignore\_error         & 2130     & 1\%     \\
language\_fix         & 11\_138  & 6\%     \\
minimal\_complete     & 151\_977 & 78\%    \\
matching\_begin       & 2848     & 1\%     \\
push\_back            & 13\_694  & 7\%     \\
undo\_reduce          & 7491     & 4\%     \\
insert                & 141\_629 & 72\%    \\
delete                & 18\_557  & 9\%     \\
string\_quote         & 2615     & 1\%     \\
\hline
                      & mean     & max     \\
Enqueue               & 796.3    & 60\_147 \\
Check                 & 61.8     & 5496  \\
\hline
\end{tabular}
\caption{Error recovery statistics for full file parsing}
\label{table:full-recover-stats}
\end{figure}

\begin{figure}[H]
\begin{tabular}{| l | r | r |}
\hline
fail enqueue limit & 16      & 0\%      \\
\hline
ignore\_error      & 850     & 5\%     \\
language\_fix      & 356     & 2\%     \\
minimal\_complete  & 14\_579 & 77\%    \\
matching\_begin    & 2355    & 13\%    \\
push\_back         & 1019    & 5\%     \\
undo\_reduce       & 544     & 3\%     \\
insert             & 6076    & 32\%    \\
delete             & 1266    & 7\%     \\
string\_quote      & 11      & 0\%     \\
\hline
                   & mean    & max     \\
Enqueue            & 922.6   & 49\_286 \\
Check              & 78.5    & 4201    \\
\hline
\end{tabular}
\caption{Error recovery statistics for partial file parsing}
\label{table:partial-recover-stats}
\end{figure}

\section{Bibliography}
\begin{thebibliography}{9}

\bibitem{Ada 2012} \verb|http://ada-auth.org/standards/ada12.html|

\bibitem{DeRemer 1982} Frank DeRemer, Thomas Pennello: Efficient
  Computation of LALR(1) Look-Ahead Sets. ACM Transactions on
  Programming Languages and Systems, Vol. 4, No. 4, October 1982,
  pages 615-649.
    % "d:/Books/eBooks/Parsing/DeRemer, Pennello 1982 Efficient Computation.pdf"

\bibitem{Emacs Ada mode news} http://www.nongnu.org/ada-mode/NEWS-ada-mode.text

\bibitem{FMQ 1980}  Fischer, C.N., Milton, D.R., Quiring, S.B.:
  Efficient LL(1) error correction and recovery using only insertions.
  Acta Inf. 13(2), 141-154 (1980)

\bibitem{graph-cycles} Donald B. Johnson, Finding all the Elementary Circuits of a Directed Graph.
SIAM J. Comput. Vol 4, No. 1, March 1975.\\ \verb|https://epubs.siam.org/doi/abs/10.1137/0204007|

\bibitem{Grune 2008} Dick Grune, Ceriel J.H. Jacobs: ``Parsing Techniques; A Practical Guide, Second
  Edition''. Springer Science  2008.

\bibitem{javarm}
  \verb|https://docs.oracle.com/javase/specs/jls/se12/html/index.html|

\bibitem{McKenzie 1995} McKenzie, Bruce J., Yeatman, Corey, and De
  Vere, Lorraine. Error repair in shift reduce parsers. ACM Trans.
  Prog. Lang. Syst., 17(4):672-689, July 1995.
  % Described in [Grune 2008] ref 321.

\bibitem{Tomita 1986} Masaru Tomita. Efficient parsing for natural language. Kluwer Academic Publishers, Boston, 1986.
\end{thebibliography}
\end{document}
